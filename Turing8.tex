
\documentclass[11pt]{article}

\usepackage[utf8]{inputenc}

\usepackage{mathrsfs}
\usepackage{amsthm}
\usepackage{pifont}
\usepackage{marvosym}
\usepackage{geometry} 
\geometry{letterpaper} 
\usepackage{amsmath}
\usepackage{graphicx} 
\usepackage[parfill]{parskip} 
\usepackage{booktabs} 
\usepackage{array}
\usepackage{paralist} 
\usepackage{verbatim} 
\usepackage{subfig} 

\renewcommand\qedsymbol{\ding{121}}


%%% HEADERS & FOOTERS
\usepackage{fancyhdr}
\pagestyle{fancy} % options: empty , plain , fancy
\renewcommand{\headrulewidth}{5pt} 
\lhead{}\chead{}\rhead{}
\lfoot{}\cfoot{\thepage}\rfoot{}

%%% SECTION TITLE APPEARANCE
\usepackage{sectsty}
%\allsectionsfont{\sffamily\mdseries\upshape} 
%%% ToC (table of contents) APPEARANCE
\usepackage[nottoc,notlof,notlot]{tocbibind} 
\usepackage[titles,subfigure]{tocloft} 
%\renewcommand{\cftsecfont}{\rmfamily\mdseries\upshape}
%\renewcommand{\cftsecpagefont}{\rmfamily\mdseries\upshape} 
%%% END Article customizations


\title{An Arbitrarily Self-validating Configuration of Turing Machines}


\author{Mark Inman, Ph.D. \\ mark.inman@egs.edu}
\date{November 21, 2016}


\begin{document}
\maketitle
\abstract{We present a top level description for the configuration of self-validating Turing machines modified with parallel inputs and based on Turing's $\mathscr{H}$  Machine as he describes in his paper, ``On Computable Numbers, with an Application to the Entscheidungsproblem." This novel description allows such a machine configuration to solve for $\beta^{'}$ by introducing a parallel computing component and memory check procedure to Turing's original conception of an  $\mathscr{H}$ Machine. Solving for $\beta^{'}$ is RE-complete and has implications for complexity class relationships.}
\section{Overview}
\subsection{Context}
This article presents a top level description for the configuration of modified $\mathscr{H}$ Machines with parallel inputs and the ability to self-verify. While this article is not entirely self-contained, we have limited our outside references of significance to just Turing's original paper, ``On Computable Numbers, with an application to the Entscheidungsproblem," which is also the only reference necessary to understand this article. %%%well, at least until the end of the penultimate proof... but even though the final proof isn't also self-contained, it's truth should be an obvious consequence of the penultamate proof. So, I'll just leave the final proof in the comments section of this paper, and the editor can choose or not choose to include it.
So, to avoid any confusions that might arise from material written since Turing's source document, and to eliminate any iota of external influence and bias, we have chosen to use original terminology and reference descriptions as they are found in Turing's paper. This means that, while we accept his work as true, we have reason to believe it is incomplete; that it does not account for the machine configuration we describe in this paper. Thus, we will consider Turing's primary results on the Entschiedungsproblem as if they have not yet been universally accepted as true for all cases until the novel case presented in this article is reviewed. If we find no discrepency between his results and ours, this will be a double confirmation of his results through our unique method employing the concept of Turing's machine. If a discrepency is found, then we should spend time to re-evaluate our notions of computability and related arguments such as Cantor's Diagonal Method and countability.

%%%this will be a double confirmation of the results of my earlier work in formal languages and Cantor's Diagonal method that I referred to in my disertation which implies a solution to $\beta^{'}$. If however, we can not solve for $\beta^{'}$, I will retract my earlier results. But since we can, I won't.

\subsection{Preliminary Considerations}
By convention, the terms $\emph{Circular Machine}$ and $\emph{Circle-free Machine}$ are not typically used, as we conveniently may refer to the halting or infinite loops of programs, where infinite loops are unsatisfactory and halting is satisfactory. However, in our circumstance, it is convenient to note here that a $\emph{Circular Machine}$ is deemed by Turing to be unsatisfactory due to it's inability to continue through it's input set of all Description Numbers as Turing defines his Universal Machine, $\mathscr{U}$. Also, a $\emph{Circle-free Machine}$ is deemed by Turing to be satisfactory based on those same definitions and an ability to continue deciding over the list of all Description Numbers. We have chosen to keep Turing's original terminology for the sake of clarity when comparing the work of this article to that of Turing's original paper. This same reasoning for our choice of nomenclature, is employed for all subsequent terminology which may no longer be used by only a matter of convention as well. \cite{Turing}

A $\emph{Standard Description}$ or S.D. is the rule set for any given Turing Machine $\mathscr{M}$ in a standard form. By creating a standard, the rule sets themselves can be used to create a $\emph{Description Number}$ or D.N. which itself may be readable by a Universal Turing Machine, $\mathscr{U}$, as an instruction set. \cite{Turing}

From Turing's paper: ``Let $\mathscr{D}$ be the Turing Machine which when supplied with the Standard Description (S.D.) of any computing machine $\mathscr{M}$ will test this S.D. and if $\mathscr{M}$ is circular will mark the S.D. with the symbol ``$\emph{u}$" and if it is circle free, will mark it with `$\emph{s}$' for `unsatisfactory' and `satisfactory' respectively. By combining machines $\mathscr{D}$ and $\mathscr{U}$, we could construct a machine $\mathscr{H}$ to compute the sequence of $\beta^{'}$" \cite{Turing}

$\mathscr{H}$ is circle free by construction and if it weren't for Turing's findings, we could accept the machine as circle free intutively and right away. However, there is nothing intuitive about the Entscheidungsproblem, nor it's solution. \cite{Turing}

The only additional prerequisite needed for this article is that the reader accepts the Church-Turing thesis that says every function which is calculable, is a computable function. It is not in the scope of this paper to re-invent the wheel and give an exhaustive description of Turing's machines. We could very well choose to use a ``black box machine'' that has the same expressive power and limitations as the Universal Turing Machine and this article should still hold. There are many online and free resources with in depth and clear descriptons, as well asvideos of working physical models of Turing Machines which are widely available for the public to view and understand their expresive power. For an in depth understanding of Turing Machines, the Universal Turing Machine and the history behind it's development, I recomend one reads $\emph{The Annotated Turing}$ by Charles Petzold. \cite{Petzold}

\subsection{Turing's Claim}
In the eighth section of Turing's paper on the Entscheidungsproblem, Turing claims that $\beta^{'}$ can not be determined vis a vis the following reason:

``The instructions for calculating the R(K)-th [figure] would amount to `calculate the first R(K)-th figures computed by $\mathscr{H}$ and write down the R(K)-th'. This R(K)-th would never be found. I.e. $\mathscr{H}$ is circular..." \cite{Turing}

This is because, since $\mathscr{H}$ relies on certain subroutines to make it's determination, when it reaches and tries to evaluate K, it must call itself, which provides instructions on reading inputs from 1 to K-1 in order to call the R(K)-th figure, but it can never get there, because it keeps repeating it's own instruction loop. \cite{Turing} The question is, however, is there a Turing Machine which can recognize itself arbitrarily\footnote{by recognizing itself arbitrarily, we mean that it can recognize it's own program or description number even if K is not fixed as the program continues to run. Also, there may not exist an initializer that feeds a fixed K to be recognized by a single read instruction that skips K and just rubber stamps approval. Such ``rubber stamping'' is considered a trivial case and is not of concern to this article.} when it reaches it's own Description Number (D.N.), such that some $\mathscr{H}$ machine configuration prints $\beta^{'}$?

\section{Self-validating Computers}
\subsection{Supermachine} %I think it's a cool name, so I'm not changing it. When i was a kid, I wanted to be like Luke Skywalker who did ``Super Flips''. I just noticed that my new LaTeX will do the correct context on my quotes automatically... ``Cool!'' I hated having to go back and correct those all the time. 


Now, let us consider that $\mathscr{H}^{'}$ is a controller machine with a D.N. of $K^{'}$. It controls two different $\mathscr{H}$ machines: $\mathscr{H}_0$ and $\mathscr{H}_1$. $\mathscr{H}_0$ and $\mathscr{H}_1$ each have the  ability to determine ``$\emph{u}$" or ``$\emph{s}$" on a D.N. input, except $\mathscr{H}_0$ tests as Turing describes, from D.N. 1 counting upwards (Each D.N. is a natural number) and $\mathscr{H}_1$ tests from a certain twos complement of whatever number is being tested by $\mathscr{H}_0$ as a simultaneous parallel input, such that it's subsequent D.N. is one less than the previously tested D.N. Let us represent each D.N. by some integer $i$. $\mathscr{H}_0$ and $\mathscr{H}_1$ have a unique D.N. of $K_0$ or $K_1$ respectively.\footnote{This can be determined through a unique identifier string, which does not effect The machine's function or performance, but differentiates the two machines from each other giving them each a unique D.N.}

Upon input of $i_0$ to be read by $\mathscr{H}_0$, let $\mathscr{H}^{'}$ store the value pair $(i_0, z)$ until $i_0$ is determined to be satisfactory or unsatisfactory. When the output is determined, let $\mathscr{H}^{'}$ replace the $(i_0, z)$ with the respective $(i_0, s)$ or $(i_0, u)$ in the data store, such that there is no longer a data store of  $(i_0, z)$. Let the same process occur for $i_1$, such that $\mathscr{H}^{'}$ also initially stores each D.N. input with $(i_1, z)$ and subsequently replaces the initial value pair with the respective value pair $(i_0, s)$ or $(i_0, u)$ depending on the output of $\mathscr{H}_1$. A \emph{redundancy} occurs when some $i_0 = i_1$. 

Let $\mathscr{H}^{'}$ have the ability to compare value pairs such that the machine may recognize a redundancy when it occurs, and may also recognize when a value pair contains a z value on the condition of such a redundancy. Let's call this a \emph{z-check} ability.

Let $\mathscr{H}_s$ be the supermachine that is the configuration of all three $\mathscr{H}$ Machines as described above and let $K_s$ be the D.N. for the supermachine.

Initialize the identifier strings such that $K_1 < K_0$. %This is arbitrary, just need to have one K have a description number that is less than the other, i.e. they can't have the exact same configurations, even though they have the same decision abilities when run by a Universal Turing Machine, and let's make sure that we call that particular D.N., K_1.

Let the number of bits in $K_0 = n$. Let the twos complement of the first D.N. input to $\mathscr{H}_0$ , which is 1, be determined by $n$ such that it satisifies the equation $c = 2^n - 1$.
\\

\noindent{\emph{Lemma}. \emph{ $K_1$ is decided by $\mathscr{H}_0$, $K_0$ is decided by $\mathscr{H}_1$, and  $\mathscr{H}_s$ proceeds taking inputs circle free, or at least until it reads $K_s$.} If $c - K_0 > K_1$, then re-initialize the D.N.\footnote{one may re-initialize, if necessay, the D.N. by adding irrelevent description information into some S.D. yielding a different D.N. provided such information does not affect the integrity of the original S.D.} %I mean, it's not necessary to do this, just choose the respective K appropriately.
in either $K_0$ or $K_1$ such that $c - K_0 < K_1$. This guarantees that $\mathscr{H}_0$ will read $K_1$ before $\mathscr{H}_1$  reads $K_1$ and also guarantees $\mathscr{H}_1$  will read $K_0$ before $\mathscr{H}_0$ reads $K_0$. Let the controller $\mathscr{H}^{'}$ contain a memory command which stores the decision value pairs of $\mathscr{H}_0$ and $\mathscr{H}_1$. The controller may routinely check for a redundancy on the next input. 

Now consider when $\mathscr{H}_0$ reads $K_1$, and $K_1$ calls the D.N. for $\mathscr{H}_0$: $\mathscr{H}_0$ will call $K_1$ and the redundancy check will kick into place, recognizing that the value pair ($K_1$, z) is already stored in memory, and therefore, since $K_1 < c$, we know that $K_1$ is the description number for the other $\mathscr{H}$ machine, which is $\mathscr{H}_1$, and we can mark it as satisfactory. This same reasoning can be applied for when  $\mathscr{H}_1$ reads $K_0$. 

If however, the machine has determined a redundancy occured on a value pair where the value is either $(i,s)$ or $(i,u)$ (i.e., the z-check is negative during a positive redundancy check), then we have already evaluated this D.N. from the other $\mathscr{H}$ machine, and we no longer have to continue within the range 1 to $c$, since they will all have been decided. The supermachine, at this point proceeds to utilize machine $\mathscr{H}_0$ and proceeds from D.N. input value $c + 1$, and continues through the rest of all Description Numbers,  $c + 2$,  $c + 3$, etc... at least until it reaches it's own D.N.,  $K_s$, for no other D.N. should be problematic in determining the output decision. Thus, $\mathscr{H}_s$ proceeds circle free, at least until it reaches $K_s$ which is easily constructed to be larger than $c$.} \qedsymbol
\\

\subsection{$\beta^{'}$ is Decidable}
\begin{proof}

 \emph{$\beta^{'}$ is Decidable.} At the point $K^{'}$ is received as an input, it is determined satisfactory by either $\mathscr{H}_0$ or $\mathscr{H}_1$. Neither $\mathscr{H}_0$ nor $\mathscr{H}_1$ are called during this phase of the process.
 
 By lemma, $K_0$ is decided by $\mathscr{H}_1$, $K_1$ is decided by $\mathscr{H}_0$ and $\mathscr{H}_s$  continues indefinitely until we reach $K_s$, which describes $\mathscr{H}_s$. $K_s$ is read by $\mathscr{H}^{'}$ and as before, it's Description Number is stored along with it's temporary pair value of z until $\mathscr{H}_0$ or $\mathscr{H}_1$ returns a value for $\beta^{'}$ at that location. $K_s$ is sent to be verified by $\mathscr{H}_0$, which when $\mathscr{H}^{'}$ calls $K_s$ for a second time, under the given recursive property of $K_s$ which will eventually call itself, the z-check for value pair $(K_s, z)$ is recognized as both redundant and with a z value, stored by $\mathscr{H}^{'}$ in the data store, but because the associated value is z,the z-check ability tells us this process has already occurred, sends $K_s$ to $\mathscr{H}_1$, which self-verifies repeated z-check values. By construction, the only value $K_i$ which can provide this multiple z-check values where $K_i > c$ is $K_s$, so $\mathscr{H}_s$ now self-verifies the input $K_s$ as it's own D.N., provides a value of ``$\emph{s}$" for satisfactory, %if that is it's opinion of itself...
 and moves to evaluate $K_s + 1$ to continue indefinitely as a \emph{Circle-free Turing Machine}. 
\\

Therefore, $\beta^{'}$ is decidable over the set of all Description Numbers given some $\mathscr{H}_s$ Machine. As a direct consequence,

...

$RE\subseteq PSPACE$ 
\end{proof}

%%%%OPTIONAL ENDINGS and MUSINGS::::::::
%As we now may understand, while true, Turing's description of $\mathscr{H}$ is particular to the case he presented, and all known cases until now, but is not universally valid for all cases. And thusly, it can be clearly stated that the Entscheidungsproblem is decidable under particular configurations, despite our historical biases which indicate the contrary. 

%\begin{rant}
%As such, I do not retract my previous results which found a non-trivial exception to Cantor's diagonal method and novel results on countability. 
%The paper is available for review, although needs a lot more work than this one does. I stand by my method and results on Cantor's Diagonal method, I just need someone to help me fine tune it.
%\end{rant}

%\begin{poem}

%%% You will laugh with incredulity at what I wrote, 
%%% But until you read this entire article written,
%%% With the utmost care to understand who's been smitten, 
%%% You will never know the beauty I have wrought.

%%% For 15 years I toiled over these mathematics,
%%% My dedication to the truth of abstraction,
%%% No professor has ever had a reaction,
%%% And so at age 37, I remain an unpublished mathematician,

%%% So before you discard the proof below
%%% I can only hope that you read all of the above,
%%% To take your time and not dismiss,
%%% Because a poetic verse, had a slight difference,

%%% So to you, distinguished man of learning,
%%% How do you answer this boy of yearning?

%\end{poem}

%%%%%~~~THE MOTHER PROOF~~~ 
%%OPTIONAL::::::
%
%And as such, with the Entschiedungsproblem being RE-complete, and since we may solve for arbitrary $\beta^{'}$ using the above configuration, it is nearly trivial to conclude:
%\\
%\\
%	\begin{proof}
%		since $PSPACE\subseteq RE$, and  
%\\
%		since any given Recursively Enumerable set is contained in PSPACE,
%\\
%		by a consequence of this article's penultimate proof, which solves for RE contained in PSPACE, 
%\\
%		$RE\subseteq PSPACE$ ...
%\\
%		$RE = PSPACE$. 
%\\
%		such that since $XPSPACE\subseteq RE$
%\\
%		and $RE = PSPACE$, implies
%\\
%		$PSPACE=XPSPACE$, implies 
%\\
%		$P = NP$
%	\end{proof}
%

%%%In all sincerity, show me where I must be wrong, so I can fix it. Otherwise, publish this damned monstrocity!

  \begin{thebibliography}{1}

  \bibitem{Petzold} Charles Petzold {\em The Annotated Turing
  }  2008: Wiley Publishing, Indianapolis, IN.
\bibitem{Turing} A. M. Turing {\em ``On Computable Numbers, with an Applcation to the Entscheidungsproblem"} 1936-1937: Proceedings of the London Mathematical Society, Series 2, 42, pp 230?265.


  \end{thebibliography}

\end{document}
